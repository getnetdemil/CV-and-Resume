%!TEX TS-program = xelatex
%!TEX encoding = UTF-8 Unicode
% Awesome CV LaTeX Template for CV/Resume
%
% This template has been downloaded from:
% https://github.com/posquit0/Awesome-CV
%
% Author:
% Getnet D. Jenberia <getnet.demil@oulu.fi>
%
% Template license:
% CC BY-SA 4.0 (https://creativecommons.org/licenses/by-sa/4.0/)
%
%-------------------------------------------------------------------------------
% CONFIGURATIONS
%-------------------------------------------------------------------------------
\documentclass[11pt, a4paper]{awesome-cv}

% Load packages
\usepackage{hyperref}
\usepackage{orcidlink}

% Configure page margins with geometry
\geometry{left=1.4cm, top=.8cm, right=1.4cm, bottom=1.8cm, footskip=.5cm}

% Specify the location of the included fonts
\fontdir[fonts/]

% Color for highlights
\colorlet{awesome}{awesome-skyblue}

% Set false if you don't want to highlight section with awesome color
\setbool{acvSectionColorHighlight}{true}

% If you would like to change the social information separator from a pipe (|) to something else
\renewcommand{\acvHeaderSocialSep}{\quad\textbar\quad}

%-------------------------------------------------------------------------------
%	PERSONAL INFORMATION
%-------------------------------------------------------------------------------
\name{Getnet Demil}{Jenberia}
\position{AI \& Computer Vision Researcher{\enskip\cdotp\enskip}PhD Candidate}
\address{University of Oulu, Finland}
\mobile{+358417142887}
\email{getnet.demil@oulu.fi}
\homepage{getnetdemil.github.io}
\github{getnetdemil}
\linkedin{getnetdemil}
\googlescholar{Xz_tNToAAAAJ}{Google Scholar}
\orcidlink{0009-0006-5158-0895}

%-------------------------------------------------------------------------------
\begin{document}

% Print the header with above personal informations
\makecvheader

% Print the footer with 3 arguments(<left>, <center>, <right>)
\makecvfooter
  {\today}
  {Getnet D. Jenberia~~~·~~~Curriculum Vitae}
  {\thepage}

%-------------------------------------------------------------------------------
%	CV/RESUME CONTENT
%-------------------------------------------------------------------------------

%---------------------------------------------------------
\cvsection{Professional Summary}
%---------------------------------------------------------
\begin{cvparagraph}
A dedicated researcher specializing in advanced image processing and computer vision techniques for hydrology and environmental sciences. Extensive experience in developing and implementing deep learning models using Python, PyTorch, and TensorFlow to address complex challenges in remote sensing and climate change analysis. Proven ability to deliver high-impact research and contribute to open-source projects.
\end{cvparagraph}

%---------------------------------------------------------
\cvsection{Degrees}
%---------------------------------------------------------
\begin{cventries}
  \cventry
    {PhD in Artificial Intelligence and Remote Sensing Technology}
    {University of Oulu}
    {Oulu, Finland}
    {2024 - Present}
    {
      \begin{cvitems}
        \item Research focus on AI-driven hydrology modeling and satellite image analysis.
      \end{cvitems}
    }
  \cventry
    {Triple Master's Degree in Image Processing and Computer Vision}
    {Erasmus Mundus Joint Master's Degree}
    {France, Spain, Hungary}
    {2022 - 2024}
    {
      \begin{cvitems}
        \item Specialized in advanced computer vision algorithms and deep learning architectures across three leading European universities.
      \end{cvitems}
    }
  \cventry
    {MSc in Communication System Engineering}
    {Bahir Dar University}
    {Bahir Dar, Ethiopia}
    {2020 - 2022}
    {}
  \cventry
    {B.Sc. in Electrical Engineering}
    {Bahir Dar University}
    {Bahir Dar, Ethiopia}
    {2013 - 2018}
    {}
\end{cventries}

%---------------------------------------------------------
\cvsection{Key Skills}
%---------------------------------------------------------
\begin{cvskills}
  \cvskill
    {Programming & Tools}
    {Python, C++, MATLAB, Git, Docker}
  \cvskill
    {Deep Learning}
    {PyTorch, TensorFlow, Keras, OpenCV}
  \cvskill
    {Technologies}
    {Computer Vision, Remote Sensing, GIS, Satellite Image Analysis}
  \cvskill
    {Languages}
    {English (Fluent), Amharic (Native), Spanish (Intermediate)}
\end{cvskills}

%---------------------------------------------------------
\cvsection{Professional Experience}
%---------------------------------------------------------
\begin{cventries}
  \cventry
    {Doctoral Researcher}
    {University of Oulu}
    {Oulu, Finland}
    {2024 - Present}
    {
      \begin{cvitems}
        \item Developing novel AI-driven methods for satellite image processing to estimate snow water characteristics.
        \item Enhancing hydrological modeling accuracy using deep learning techniques for water resource management.
      \end{cvitems}
    }
  \cventry
    {Computer Vision Research Associate}
    {University of Oulu}
    {Oulu, Finland}
    {2024}
    {
      \begin{cvitems}
        \item Established a data processing pipeline for hydrological parameter estimation from image sensors.
        \item Achieved state-of-the-art accuracy in snow classification using a modified DeepLabV3+ architecture.
      \end{cvitems}
    }
  \cventry
    {Junior Electrical Engineer}
    {Ethiopian Electric Utility}
    {Ethiopia}
    {2018 - 2021}
    {
      \begin{cvitems}
        \item Contributed to the design and maintenance of electrical power systems and national infrastructure.
      \end{cvitems}
    }
  \cventry
    {Chief Technology Support}
    {American Space Ethiopia (U.S. Embassy)}
    {Ethiopia}
    {2019 - 2020}
    {
      \begin{cvitems}
        \item Managed IT infrastructure and provided technology support for a U.S. Embassy public diplomacy initiative.
      \end{cvitems}
    }
\end{cventries}

%---------------------------------------------------------
\cvsection{Research Output}
%---------------------------------------------------------
\begin{cventries}
  \cventry
    {Journal of Hydrology}
    {Advances in image-based estimation of snow variable: A systematic literature review on recent studies}
    {}
    {2024}
    {
      \begin{cvitems}
        \item DOI: \href{https://doi.org/10.1016/j.jhydrol.2025.132855}{10.1016/j.jhydrol.2025.132855}
      \end{cvitems}
    }
  \cventry
    {Earth Science Informatics}
    {Seeing through the clouds: enhanced snow and cloud segmentation in Sentinel-2 imagery with mDeepLabV3+}
    {}
    {2025 (In Press)}
    {
      \begin{cvitems}
        \item DOI: \href{https://doi.org/10.1007/s12145-025-01950-6}{10.1007/s12145-025-01950-6}
      \end{cvitems}
    }
  \cventry
    {Book Chapter (arXiv)}
    {AI-based Approach in Early Warning Systems: Focus on Emergency Communication Ecosystem and Citizen Participation in Nordic Countries}
    {}
    {2025}
    {
      \begin{cvitems}
        \item Preprint: \href{https://arxiv.org/abs/2506.18926}{arXiv:2506.18926}
      \end{cvitems}
    }
  \cventry
    {SIGIR 2024 Workshop}
    {Leveraging Social Media for Real-time Monitoring of Local Climate Impact}
    {Washington DC, USA}
    {2024}
    {
      \begin{cvitems}
        \item DOI: \href{https://doi.org/10.48550/arXiv.2504.01162}{10.48550/arXiv.2504.01162}
      \end{cvitems}
    }
  \cventry
    {Nordic Workshop on AI for Climate Change}
    {AI-Enhanced Snow and Cloud Segmentation in Sentinel-2 Imagery}
    {Copenhagen, Denmark}
    {2025}
    {}
\end{cventries}

%---------------------------------------------------------
\cvsection{Awards and Honours}
%---------------------------------------------------------
\begin{cvhonors}
  \cvhonor
    {Best 50 African Project of the Year}
    {Africa Innovation Week}
    {Continental}
    {2019}
  \cvhonor
    {Best Bahir Dar University Project of the Year}
    {Bahir Dar University}
    {Ethiopia}
    {2018}
\end{cvhonors}

%---------------------------------------------------------
\cvsection{References}
%---------------------------------------------------------
\begin{cvparagraph}
Available upon request.
\end{cvparagraph}

\end{document}
