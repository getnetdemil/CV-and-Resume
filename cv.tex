%!TEX TS-program = xelatex
%!TEX encoding = UTF-8 Unicode
% Awesome CV LaTeX Template for CV/Resume
%
% This template has been downloaded from:
% https://github.com/posquit0/Awesome-CV
%
% Author:
% Getnet D. Jenberia <getnet.demil@oulu.fi>
% Based on Awesome-CV template
%
% Template license:
% CC BY-SA 4.0 (https://creativecommons.org/licenses/by-sa/4.0/)
%
%-------------------------------------------------------------------------------
% CONFIGURATIONS
%-------------------------------------------------------------------------------
% A4 paper size by default, use 'letterpaper' for US letter
\documentclass[15pt, a4paper]{awesome-cv}

% Configure page margins with geometry
\geometry{left=1.4cm, top=.8cm, right=1.4cm, bottom=1.8cm, footskip=.5cm}

% Specify the location of the included fonts
\fontdir[fonts/]

% Color for highlights
% Awesome Colors: awesome-emerald, awesome-skyblue, awesome-red, awesome-pink, awesome-orange
%                 awesome-nephritis, awesome-concrete, awesome-darknight
\colorlet{awesome}{awesome-skyblue}

% Set false if you don't want to highlight section with awesome color
\setbool{acvSectionColorHighlight}{true}

% If you would like to change the social information separator from a pipe (|) to something else
\renewcommand{\acvHeaderSocialSep}{\quad\textbar\quad}

%-------------------------------------------------------------------------------
%	PERSONAL INFORMATION
%	Comment any of the lines below if they are not required
%-------------------------------------------------------------------------------
\name{Getnet D.}{Jenberia}
\position{AI \& Computer Vision Researcher{\enskip\cdotp\enskip}PhD Candidate}
\address{University of Oulu, Finland}
\mobile{+358417142887}
\email{getnet.demil@oulu.fi}
\homepage{getnetdemil.github.io}
\github{getnetdemil}
\linkedin{getnetdemil}
\quote{``Advancing hydrology modeling through artificial intelligence and remote sensing technology."}

%-------------------------------------------------------------------------------
\begin{document}

% Print the header with above personal informations
\makecvheader

% Print the footer with 3 arguments(<left>, <center>, <right>)
\makecvfooter
  {\today}
  {Getnet D. Jenberia~~~·~~~Curriculum Vitae}
  {\thepage}

%-------------------------------------------------------------------------------
%	CV/RESUME CONTENT
%-------------------------------------------------------------------------------

%---------------------------------------------------------
\cvsection{Profile}
%---------------------------------------------------------
\begin{cvparagraph}

A dedicated researcher specializing in advanced image processing and computer vision techniques applied to hydrology modeling. With extensive experience in deep learning methodologies, I tackle complex problems in hydrology and environmental sciences using Python, PyTorch, and cutting-edge AI frameworks. My research focuses on leveraging satellite imagery and remote sensing data to enhance water resource management and climate change adaptation strategies.
\end{cvparagraph}

%---------------------------------------------------------
\cvsection{Education}
%---------------------------------------------------------
\begin{cventries}
  \cventry
    {PhD in Artificial Intelligence and Remote Sensing Technology} % Degree
    {University of Oulu} % Institution
    {Oulu, Finland} % Location
    {2024 - Present} % Date(s)
    {
      \begin{cvitems} % Description(s) bullet points
        \item {Focus: Hydrology Modeling using AI and Remote Sensing}
        \item {Research: Deep learning for snow water characteristics estimation from satellite imagery}
        \item {Applications: Water resource management and climate change strategies}
      \end{cvitems}
    }

  \cventry
    {Triple Master's Degree in Image Processing and Computer Vision} % Degree
    {Erasmus Mundus Joint Master's Degree } % Institution
    {France, Spain, Hungary} % Location
    {2022 - 2024} % Date(s)
    {
      \begin{cvitems} % Description(s) bullet points
        \item {University of Bordeaux, France}
        \item {Autonomous University of Madrid, Spain}
        \item {Pazmany Peter Catholic University, Hungary}
        \item {Specialized in advanced computer vision algorithms and deep learning architectures}
      \end{cvitems}
    }

  \cventry
    {MSc in Communication System Engineering} % Degree
    {Bahir Dar University} % Institution
    {Bahir Dar, Ethiopia} % Location
    {2020 - 2022} % Date(s)
    {}

  \cventry
    {B.Sc. in Electrical Engineering} % Degree
    {Bahir Dar University} % Institution
    {Bahir Dar, Ethiopia} % Location
    {2013 - 2018} % Date(s)
    {
      \begin{cvitems}
        \item {Major: Electronics and Communication Systems}
      \end{cvitems}
    }
\end{cventries}

%---------------------------------------------------------
\cvsection{Research Experience}
%---------------------------------------------------------
\begin{cventries}
  \cventry
    {Doctoral Researcher} % Job title
    {University of Oulu} % Organization
    {Oulu, Finland} % Location
    {09.2024 - Present} % Date(s)
    {
      \begin{cvitems} % Description(s) of tasks/responsibilities
        \item {Developing AI-driven methods for satellite image processing to determine snow water characteristics}
        \item {Enhancing hydrological modeling accuracy using deep learning techniques}
        \item {Applying PyTorch and TensorFlow for environmental parameter estimation}
        \item {Integrating multi-modal sensing data (Sentinel-1, Sentinel-2, DEMs, reanalysis data)}
      \end{cvitems}
    }

  \cventry
    {Computer Vision Research Intern} % Job title
    {University of Oulu} % Organization
    {Oulu, Finland} % Location
    {02.2024 - 07.2024} % Date(s)
    {
      \begin{cvitems} % Description(s) of tasks/responsibilities
        \item {Established comprehensive data processing pipelines for hydrological parameter estimation}
        \item {Achieved breakthrough accuracy in snow classification using modified DeepLabV3+ architecture}
        \item {Developed novel deep learning models for snow-cloud segmentation in satellite imagery}
      \end{cvitems}
    }
\end{cventries}

%---------------------------------------------------------
% \cvsection{Other Work Experience}
% %---------------------------------------------------------
% \begin{cventries}
%   \cventry
%     {Junior Electrical Engineer} % Job title
%     {Ethiopian Electric utility} % Organization
%     {Ethiopia} % Location
%     {3.5 Years} % Date(s)
%     {
%   Full-time job
%     }

%   \cventry
%     {Chief Technology Support} % Job title
%     {American Space Ethiopia} % Organization
%     {Ethiopia} % Location
%     {1.5 Years} % Date(s)
%     {
%       American Embassy in Ethiopia
%     }
% \end{cventries}

\cvsection{Skills}
%---------------------------------------------------------
\begin{cvskills}
  \cvskill
    {Programming} % Category
    {Python, C++, MATLAB, JavaScript} % Skills

  \cvskill
    {Deep Learning} % Category
    {PyTorch, TensorFlow, Keras, OpenCV} % Skills

  \cvskill
    {Computer Vision} % Category
    {Image Processing, Object Detection, Semantic Segmentation, 6D Pose Estimation} % Skills

  \cvskill
    {Remote Sensing} % Category
    {GIS, Google Earth Engine, Satellite Image Analysis, Multi-spectral Processing} % Skills

  \cvskill
    {Systems} % Category
    {Linux (Ubuntu, Debian), Windows, Git, Docker} % Skills

  \cvskill
    {Languages} % Category
    {English (Fluent), Amharic (Native), Spanish (Intermediate)} % Skills
\end{cvskills}

%---------------------------------------------------------
\cvsection{Publications}
%---------------------------------------------------------
\begin{cventries}
  \cventry
    {Journal of Hydrology} % Journal
    {Advances in image-based estimation of snow variable: A systematic literature review on recent studies} % Title
    {} % Location
    {June 2024} % Date
    {
      \begin{cvitems}
        \item {Comprehensive review emphasising image-based deep learning architectures for snow hydrology modelling}
        \item {DOI: 10.1016/j.jhydrol.2025.132855}
      \end{cvitems}
    }

  \cventry
    {Earth Science Informatics} % Journal
    {Seeing through the clouds: enhanced snow and cloud segmentation in Sentinel-2 imagery with mDeepLabV3+} % Title
    {} % Location
    {July 2025 } % Date
    {
      \begin{cvitems}
        \item {Novel deep learning model for accurate snow-cloud segmentation in satellite imagery}
        \item {DOI: 10.1007/s12145-025-01950-6}
      \end{cvitems}
    }

   \cventry
    {Book Chapter: arXiv preprint arXiv:2506.18926} % Journal
    {AI-based Approach in Early Warning Systems: Focus on Emergency Communication Ecosystem and Citizen Participation in Nordic Countries} % Title
    {} % Location
    {2025} % Date
    {
      \begin{cvitems}
       \item {DOI: arXiv:2506.18926}
      \end{cvitems}
    }
\end{cventries}

%---------------------------------------------------------
\cvsection{Conference Presentations}
%---------------------------------------------------------
\begin{cventries}
  \cventry
    {SIGIR 2024 Workshop on Information Retrieval for Climate Impact} % Conference
    {Leveraging Social Media for Real-time Monitoring of Local Climate Impact} % Title
    {Washington DC, USA} % Location
    {2024} % Date
    {
      \begin{cvitems}
        \item {DOI: 10.48550/arXiv.2504.01162}
      \end{cvitems}
    }

  \cventry
    {Nordic Workshop on AI for Climate Change} % Conference
    {AI-Enhanced Snow and Cloud Segmentation in Sentinel-2 Imagery Using Dilated DeepLabv3+ with ResNet Backbone} % Title
    {Copenhagen, Denmark} % Location
    {2025} % Date
    {
      \begin{cvitems}
        \item {High accuracy snow classification outpacing traditional methods}
      \end{cvitems}
    }
\end{cventries}

%---------------------------------------------------------
\cvsection{Selected Projects}
%---------------------------------------------------------
\begin{cventries}
  \cventry
    {AI-Enhanced Snow and Cloud Segmentation in Sentinel-2 Imagery} % Project
    {University of Oulu | Nordic Workshop on AI for Climate Change} % Organization
    {} % Location
    {2025} % Date
    {
      \begin{cvitems}
        \item {Modified DeepLabv3+ with ResNet backbone for semantic segmentation of Sentinel-2 satellite imagery}
        \item {Integrated multi-source data fusion: Sentinel-1, DEMs, and reanalysis data for enhanced snow classification}
        \item {Achieved state-of-the-art accuracy in snow-cloud discrimination, outperforming existing methods}
        \item {Open-source high-resolution snow estimation pipeline supporting climate resilience and runoff prediction}
      \end{cvitems}
    }
  \cventry
    {Deep Learning Based Snow Hydrology Parameter Estimation} % Project
    {University of Oulu | Published in Journal of Hydrology} % Organization
    {} % Location
    {2024 - 2025} % Date
    {
      \begin{cvitems}
        \item {Comprehensive systematic literature review on image-based snow parameter estimation (DOI: 10.1016/j.jhydrol.2025.132855)}
        \item {Developed deep learning model for snow-cloud segmentation using satellite imagery (DOI: 10.1007/s12145-025-01950-6)}
        \item {End-to-end pipeline: data acquisition, model selection, training methodologies, and performance evaluation}
        \item {Advanced uncertainty quantification techniques for improved hydrological modeling accuracy}
      \end{cvitems}
    }
  \cventry
    {Climate Change Impact Monitoring Using Social Media Analytics} % Project
    {Research Publication | arXiv} % Organization
    {} % Location
    {2024} % Date
    {
      \begin{cvitems}
        \item {Real-time local climate change impact assessment through social media data analysis}
        \item {Natural language processing and sentiment analysis for community-level environmental insights}
        \item {Published methodology for leveraging citizen-reported climate experiences (arXiv:2504.01162)}
        \item {Bridge between traditional climate monitoring and grassroots environmental observations}
      \end{cvitems}
    }
  \cventry
    {Vision Aided Recognition of Objects for Assistive Robotics} % Project
    {European Consortium Project} % Organization
    {} % Location
    {2022 - 2023} % Date
    {
      \begin{cvitems}
        \item {6D pose estimation using DenseFusion model for individuals with upper-limb disabilities}
        \item {Created proprietary dataset using Unity engine and HTC VIVE headsets with precise calibration}
        \item {RGB-D sensor fusion with binary mask integration for robust object manipulation}
        \item {Complete pipeline including 3D modeling, intrinsic/extrinsic parameter estimation, and real-time processing}
      \end{cvitems}
    }
  \cventry
    {Advanced Semantic Segmentation for Precision Agriculture} % Project
    {Academic Research} % Organization
    {} % Location
    {2023} % Date
    {
      \begin{cvitems}
        \item {Comparative analysis of U-Net, Attention U-Net, and DeepLabV3+ for agricultural applications}
        \item {Automated multi-class segmentation: crop fields, background, and weed classification}
        \item {UAV-based image processing pipeline optimized for precision agriculture workflows}
        \item {Performance benchmarking across different deep learning architectures}
      \end{cvitems}
    }
  \cventry
    {Smart Microscope for Automated Medical Diagnosis} % Project
    {Bahir Dar University} % Organization
    {} % Location
    {2018} % Date
    {
      \begin{cvitems}
        \item {Automatic protozoan disease detection and diagnosis using computer vision and machine learning}
        \item {End-to-end system design: hardware integration, image acquisition, and diagnostic algorithms}
        \item {\textbf{Won Best African Project Award 2019} for innovation in medical technology}
        \item {Python-based implementation with real-time processing capabilities}
      \end{cvitems}
    }
\end{cventries}

%---------------------------------------------------------
\cvsection{Honors \& Awards}
%---------------------------------------------------------
\begin{cvhonors}
  \cvhonor
    {Best 50 African Project of the Year} % Award
    {Africa Innovation Week} % Event
    {Continental} % Location
    {2019} % Date(s)

  \cvhonor
    {Best Bahir Dar University Project of the Year} % Award
    {Bahir Dar University} % Event
    {Ethiopia} % Location
    {2018} % Date(s)
\end{cvhonors}

%-------------------------------------------------------------------------------
\cvsection{Reference}
%---------------------------------------------------------
\begin{cvhonors}
  \cvhonor
    {Prof. Mourad Oussalah} % Award
    {mourad.oussalah@oulu.fi} % Event
    {University of Oulu} % Location
    {1.} % Date(s)

\end{cvhonors}
%---------------------------------------------------------

\end{document}